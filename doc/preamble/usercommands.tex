% For Feynman slashes
\usepackage{slashed}

\newcommand{\husk}[1]{\color{red}#1\color{black}}
\newcommand{\expect}[1]{\left\langle{#1}\right\rangle}
\newcommand{\Tr}{\mathrm{Tr}}
\newcommand{\tr}{\mathrm{tr}}
\newcommand{\dd}{\mathrm{d}}
\newcommand{\e}{\mathrm{e}}
\newcommand{\SU}{\mathrm{SU}}
\newcommand{\conj}[1]{\overline{#1}}
\newcommand{\sign}{\mathrm{sign}}
\newcommand{\epsrnd}{\varepsilon_\mathrm{rnd}}

% For fermi units
\newcommand{\fm}{\mathrm{fm}}

% Complex number notation
\renewcommand{\Re}{\operatorname{Re}}
\renewcommand{\Im}{\operatorname{Im}}

% Commands for writing shorthand lattice operators
\newcommand{\MU}{\hat{\mu}}
\newcommand{\NU}{\hat{\nu}}

\renewcommand{\vec}[1]{\boldsymbol{\mathbf{#1}}}
% BEGIN COMMANDS FROM LATEX FOR PHYSICISTS
% http://www.dfcd.net/articles/latex/latex.html
\newcommand{\ket}[1]{\left| #1 \right\rangle} % for Dirac bras
\newcommand{\bra}[1]{\left\langle #1 \right|} % for Dirac kets
\newcommand{\bket}[1]{\right| #1 \right\rangle} % for Dirac bras
\newcommand{\bbra}[1]{\left\langle #1 \left|} % for Dirac kets

% Front page
\setbeamertemplate{title page}{
    \begin{minipage}[c][\paperheight]{\textwidth}
        \ifx\inserttitlegraphic\@empty\else\usebeamertemplate*{title graphic}\fi
        \vfill%
        {
        \centering
        \ifx\inserttitle\@empty\else\usebeamertemplate*{title}\fi
        \ifx\insertsubtitle\@empty\else\usebeamertemplate*{subtitle}\fi
        }
        \usebeamertemplate*{title separator}
        \begin{minipage}[t]{.5\textwidth}
            \ifx\beamer@shortauthor\@empty\else\usebeamertemplate*{author}\fi
            \ifx\insertdate\@empty\else\usebeamertemplate*{date}\fi
        \end{minipage}
        \begin{minipage}[t]{.5\textwidth}
            \vspace*{2em}
            {\hspace{1.2em}\small Supervisor: \textit{Andrea Shindler} \par}
            \vspace*{0.2em}
            {\hspace{1.2em}\small Co-supervisor: \textit{Morten Hjorth-Jensen}}
        \end{minipage}%

        \begin{minipage}[t]{\textwidth}
            \centering
            \ifx\insertinstitute\@empty\else\usebeamertemplate*{institute}\fi
        \end{minipage}
        \vfill
        \vspace*{1mm}
    \end{minipage}
}



\newcommand{\CC}{C\nolinebreak\hspace{-.05em}\raisebox{.4ex}{\tiny\bf +}\nolinebreak\hspace{-.10em}\raisebox{.4ex}{\tiny\bf +}}
\def\CC{{C\nolinebreak[4]\hspace{-.05em}\raisebox{.4ex}{\tiny\bf ++ }}}

% For aligned undersets
\usepackage{stackengine}
\def\stacktype{L}
\setstackgap{L}{1.2\normalbaselineskip}
\stackMath