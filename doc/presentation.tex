% \documentclass[10pt,handout,show notes on second screen]{beamer}
\documentclass[10pt,show notes on second screen]{beamer}


\usepackage{appendixnumberbeamer}
\usepackage[numbers,sort&compress,authoryear]{natbib}
\usepackage{amsfonts}
\usepackage{amsmath}
\usepackage{mathtools}
\usepackage{amssymb}
\usepackage{mathrsfs}
\usepackage{units}

% For LatViz animations
\usepackage{animate}
% \usepackage{multimedia}


% For nice algoritms
\usepackage{algorithm}
\usepackage{algorithmicx}
\usepackage{algpseudocode}

% For coloring equations
\usepackage{color}
\makeatletter
\def\mathcolor#1#{\@mathcolor{#1}}
\def\@mathcolor#1#2#3{%
\protect\leavevmode\begingroup\color#1{#2}#3\endgroup}
\makeatother

% % For proper referencing
% \usepackage{varioref}
% \usepackage{hyperref}
% \hypersetup{
%     breaklinks={true},
% }
% \usepackage{cleveref}

% For writing C++ in listings
\usepackage{listings}
\lstset{language=C++,
        basicstyle=\ttfamily,
        keywordstyle=\color{blue}\ttfamily,
        stringstyle=\color{red}\ttfamily,
        commentstyle=\color{green}\ttfamily,
        morecomment=[l][\color{magenta}]{\#}
}

% For including sub figures
\usepackage{subcaption}

% For nice tables
\usepackage{booktabs}

% For figures and graphics'n stuff
\usepackage{graphicx}
\usepackage{caption}


% \usetheme[progressbar=frametitle]{metropolis}
\usetheme{metropolis}

% For notes
\usepackage{pgfpages}
\setbeamertemplate{note page}[plain]
% \setbeameroption{show notes on second screen=right}
\setbeameroption{hide notes}

\usefonttheme[onlymath]{serif}

% Titles, date, ect.
\title{Solving \texorpdfstring{$\SU(3)$}{SU3} Yang-Mills theory on the lattice: a calculation of selected gauge observables with gradient flow}
\date{04.07.19}
\author{Hans Mathias Mamen Vege}
\institute{University of Oslo}

\bibliographystyle{plainnat}

% Loads command list
% For Feynman slashes
\usepackage{slashed}

\newcommand{\husk}[1]{\color{red}#1\color{black}}
\newcommand{\TODO}[1]{\color{red}#1\color{black}}
\newcommand{\expect}[1]{\left\langle{#1}\right\rangle}
\newcommand{\Tr}{\mathrm{Tr}}
\newcommand{\tr}{\mathrm{tr}}
\newcommand{\dd}{\mathrm{d}}
\newcommand{\e}{\mathrm{e}}
\newcommand{\SU}{\mathrm{SU}}
\newcommand{\conj}[1]{\overline{#1}}
\newcommand{\sign}{\mathrm{sign}}
\newcommand{\epsrnd}{\varepsilon_\mathrm{rnd}}

% For fermi units
\newcommand{\fm}{\mathrm{fm}}

% Complex number notation
\renewcommand{\Re}{\operatorname{Re}}
\renewcommand{\Im}{\operatorname{Im}}

% Commands for writing shorthand lattice operators
\newcommand{\MU}{\hat{\mu}}
\newcommand{\NU}{\hat{\nu}}

\renewcommand{\vec}[1]{\boldsymbol{\mathbf{#1}}}
% BEGIN COMMANDS FROM LATEX FOR PHYSICISTS
% http://www.dfcd.net/articles/latex/latex.html
\newcommand{\ket}[1]{\left| #1 \right\rangle} % for Dirac bras
\newcommand{\bra}[1]{\left\langle #1 \right|} % for Dirac kets
\newcommand{\bket}[1]{\right| #1 \right\rangle} % for Dirac bras
\newcommand{\bbra}[1]{\left\langle #1 \left|} % for Dirac kets

% Front page
\setbeamertemplate{title page}{
    \begin{minipage}[c][\paperheight]{\textwidth}
        \ifx\inserttitlegraphic\@empty\else\usebeamertemplate*{title graphic}\fi
        \vfill%
        {
        \centering
        \ifx\inserttitle\@empty\else\usebeamertemplate*{title}\fi
        \ifx\insertsubtitle\@empty\else\usebeamertemplate*{subtitle}\fi
        }
        \usebeamertemplate*{title separator}
        \begin{minipage}[t]{.4\textwidth}
            \ifx\beamer@shortauthor\@empty\else\usebeamertemplate*{author}\fi
            \ifx\insertdate\@empty\else\usebeamertemplate*{date}\fi
        \end{minipage}
        \begin{minipage}[t]{.6\textwidth}
            \vspace*{2em}
            {\hspace{1.2em}\small Supervisor: \textit{Andrea Shindler} \par}
            \vspace*{0.2em}
            {\hspace{1.2em}\small Co-supervisor: \textit{Morten Hjorth-Jensen}}
        \end{minipage}%

        \begin{minipage}[t]{\textwidth}
            \centering
            \ifx\insertinstitute\@empty\else\usebeamertemplate*{institute}\fi
        \end{minipage}
        \vfill
        \vspace*{1mm}
    \end{minipage}
}



\newcommand{\CC}{C\nolinebreak\hspace{-.05em}\raisebox{.4ex}{\tiny\bf +}\nolinebreak\hspace{-.10em}\raisebox{.4ex}{\tiny\bf +}}
\def\CC{{C\nolinebreak[4]\hspace{-.05em}\raisebox{.4ex}{\tiny\bf ++ }}}

% For aligned undersets
\usepackage{stackengine}
\def\stacktype{L}
\setstackgap{L}{1.2\normalbaselineskip}
\stackMath
\tabularnewline

\begin{document}

\setbeamercolor{background canvas}{bg=white}
\maketitle

% \begin{frame}{Structure}
%   % \setbeamertemplate{section in toc}[sections numbered]
%   \tableofcontents[hideallsubsections]
% \end{frame}

% \begin{frame}{Structure}
%     \begin{itemize}[<+->]
%         \item Quantum Chromodynamics(QCD).
%         \item Lattice QCD.
%         \item Gradient flow.
%         \item Developing a code for solving $\SU(3)$ Yang-Mills theory.
%         \item Results.
%     \end{itemize}
%     \note{
%     \begin{itemize}[<+->]
%         \item \textbf{QCD.} We will go through and explain what QCD as well as motivate its existence.
%         \item \textbf{LQCD.} We will briefly show how one discretise the lattice and perform calculations on it.
%         \item \textbf{Gradient flow.} We will quickly introduce gradient flow and explain its effects.
%         \item \textbf{GLAC.} Will briefly present the code which we developed as well as some benchmarks. We will also present the Metropolis algorithm.
%         \item \textbf{Results.} We will present the results obtained from pure gauge calculations.
%     \end{itemize}
%     }
% \end{frame}

\begin{frame}{The four forces of nature}
\begin{columns}
    \column{0.5\textwidth}
    \begin{minipage}[b][0.45\textheight][b]{\linewidth}
        \onslide<1->{
        \begin{figure}
            \centering
            \includegraphics[width=0.6\linewidth]{../figures/gravity.jpg}
            \captionsetup{labelformat=empty}
            \caption{Gravity}
        \end{figure}}
    \end{minipage}
    \column{0.5\textwidth}
    \begin{minipage}[b][0.45\textheight][b]{\linewidth}
        \onslide<2->{
        \begin{figure}
            \centering
            \includegraphics[width=0.6\linewidth]{../figures/lightning.jpg}
            \captionsetup{labelformat=empty}
            \caption{Electromagnetism}
        \end{figure}}
    \end{minipage}
\end{columns}
\begin{columns}
    \column{0.5\textwidth}
    \begin{minipage}[b][0.45\textheight][b]{\linewidth}
        \onslide<3->{
        \begin{figure}
            \centering
            \includegraphics[width=0.6\linewidth]{../figures/weakforce_pdf}
            \captionsetup{labelformat=empty}
            \caption{Weak nuclear force}
        \end{figure}}
    \end{minipage}
    \column{0.5\textwidth}
    \begin{minipage}[b][0.45\textheight][b]{\linewidth}
        \onslide<4->{
        \begin{figure}
            \centering
            \includegraphics[width=0.6\linewidth]{../figures/nucleon/nucleon.pdf}
            \captionsetup{labelformat=empty}
            \caption{Strong nuclear force}
        \end{figure}}
    \end{minipage}
\end{columns}
\end{frame}

\begin{frame}{What is the strong force?}
Consists of:
\begin{itemize}
    \item <1->{6 quark flavors: up, down, strange, charm, bottom and top}
    \item <2->{8 gluons}
\end{itemize}%
\begin{columns}
    \column{0.5\textwidth}
    \begin{minipage}[b][0.45\textheight][b]{\linewidth}%
        \onslide<4->{A \textbf{proton} consists of: up-, up- and down-quarks\\}%
        \onslide<5->{Mass discrepancy:%
        \begin{align*}
            m_p &\neq m_u + m_u + m_d, \\
            936 \text{ MeV} &\neq 3\text{ MeV} + 3\text{ MeV} + 6\text{ MeV}.
        \end{align*}}%
    \end{minipage}
    \column{0.5\textwidth}
    \begin{minipage}[b][0.45\textheight][b]{\linewidth}
        \onslide<4->{\begin{center}
            \includegraphics[width=0.8\textwidth]{../figures/Proton_quark_structure_pdf.pdf}
        \end{center}}
    \end{minipage}
\end{columns}
\note{The mass discrepancy is due to the interaction energy in which gluons are mediators.}
\end{frame}

\begin{frame}{Comparing the strong force and QED}
\onslide<1->{\textbf{Electromagnetism} or \textbf{Quantum Electrodynamics}(QED), a U$(1)$ symmetry theory:
\begin{align*}
\mathcal{L}_\mathrm{QED} = \sum_{f=\text{fermions}}\bar{\psi}_f (i\gamma^\mu(\partial_\mu + ieQ_f A_\mu) - m_f) \psi_f - \frac{1}{4}F_{\mu\nu} F^{\mu\nu}
\end{align*}%
Field strength tensor:
\begin{align*}
F_{\mu\nu} = \partial_\mu A_\nu - \partial_\nu A_\mu
\end{align*}}%
\onslide<2->{The \textbf{strong nuclear force} or \textbf{Quantum Chromo Dynamics}(QCD), a $\SU(3)$ symmetry theory:%
\begin{align*}
\mathcal{L}_\mathrm{QCD} = \sum_{f=u, d, \dots} \bar{\psi}_f (i\gamma^\mu(\partial_\mu + ig_S A^a_\mu T^a) - m_{f}) \psi_{f} - \frac{1}{4}G^a_{\mu\nu} {G^a}^{\mu\nu} 
\end{align*}%
Field strength tensor:
\begin{align*}
G_{\mu\nu}^a = \partial_\mu A_\nu^a - \partial_\nu A_\mu^a - g_S f^{abc} A_\mu^b A_\nu^c
\end{align*}}%
\note{\begin{itemize}
    \item <1->QED is U(1) theory, and the simply means that it contains a simple rotational symmetry.
    \item <1->$e$ is the coupling constant and $Q_f$ is the charge.
    \item <1->$\bar{\psi}_f$ and $\psi_f$ are is the fermion fields.
    \item <1->$A_\mu$ is the gauge fields.
    \item <1->$\gamma^\mu$ is the Dirac matrices.
    \item <1->First part contains the dynamics of all the fermions such as electrons, muons ect.
    \item <1->Second part contains the dynamics of the photon fields in terms of the field strength tensor.
    \item <2->Instead of a sum over all fermions, it is a sum over all quarks.
    \item <2->The coupling constant is now the strong coupling, $g_S$.
    \item <2->$A_\mu^a$ is now the gluon fields, and $T^a$ is the matrices for the $\SU(3)$ gauge symmetry group.
    \item <2->QCD contains more degrees of freedom due to this.
    \item <2->The difference between U(1) and SU(2) is that the latter contains a more complex rotational symmetry.
    \item <2->The field strength tensor becomes nonlinear due to the extra term.
\end{itemize}}
\end{frame}

\begin{frame}{Why is the strong force strong?}
\begin{columns}
    \column{0.5\textwidth}
    \begin{minipage}[b][0.5\textheight][b]{\linewidth}
    % \begin{center}
    %     \includegraphics[width=0.9\textwidth]{../figures/pdg_asymptotic_freedom-eps-converted-to.pdf}
    % \end{center}
    \begin{center}
        \includegraphics[width=0.95\textwidth]{../figures/running_coupling_qed_qcd_png.png}
    \end{center}
    \tiny{\url{https://www-cdf.fnal.gov/~group/WORK/DISS_PAGE/diss_page.htm}}
    \end{minipage}
    \column{0.5\textwidth}%
    \begin{minipage}[b][0.75\textheight][b]{\linewidth}%
        \begin{itemize}
            \item \onslide<1->{Coupling constant $\alpha$ is the strength of the force in an interaction.}%
        % \begin{itemize}
        %     \item \onslide<2->{$\alpha_G \approx 1.75 \times 10^{-45}$}
        %     \item \onslide<3->{$\alpha_W \approx 10^{-6} - 10^{-7}$}
        %     \item \onslide<4->{$\alpha_\mathrm{QED} \approx \frac{1}{137} \approx 0.0073$}
        %     \item \onslide<5->{$\alpha_S \approx \frac{g_S^2}{4\pi} \approx 1$}
        % \end{itemize}
            \item \onslide<2->{QED becomes stronger - QCD becomes weaker at higher energies.}%
            \item \onslide<3->{Can't use perturbation theory on strong force in low-energy regime!}%
            \item \onslide<4->{Need to understand the low-energy regime to understand phenomenas such as \textbf{confinement}.}%
        \end{itemize}
    \end{minipage}
\end{columns}
\note{
\begin{itemize}
    \item <1->{In physics, \textbf{a coupling constant} or gauge coupling parameter (or, more simply, a coupling), is a number that determines the strength of the force exerted in an interaction.}
    % \item <5->{\textbf{Couplings at low energy.}}
    \item <2->{We cannot make \textbf{perturbative expansions} in the low-energy regime of the strong nuclear force.}
    \item <4->{Which is really a shame, since many interesting phenomena such as \textbf{confinement} is a low-energy phenomena.}
\end{itemize}}
\end{frame}

\begin{frame}{Confinement: a low-energy phenomena}
No free quarks in nature!
\begin{center}
    \includegraphics[width=0.65\textwidth]{../figures/illustrations/qcd/confinement/string-breaking.pdf}
\end{center}
\note{If we try to pull apart \textbf{two quarks in a meson}, more and more energy is required until we have enough energy to spontaneously create a \textbf{quark-antiquark} pair, forming thus \textbf{two new mesons.}}
\end{frame}

\begin{frame}{QCD and nuclear physics}
\begin{columns}
    \column{0.5\textwidth}
    \begin{minipage}[b][1.0\textheight][b]{\linewidth}
        \begin{figure}
            \centering
            \includegraphics[width=0.8\linewidth]{../figures/scales_frib.pdf}
            \captionsetup{labelformat=empty}
            \caption{\tiny\textit{FRIB Users Organization for the NSAC Long Range Plan Implementation Subcommittee}}
        \end{figure}
    \end{minipage}
    \column{0.5\textwidth}
    % {Few parameters give rise to observed phenomena: $g_S, m_d, m_u, \dots e$} \\
    \begin{itemize}
        \item \onslide<1->{Need to understand the low-energy regime in order to better understand nuclear physics!}\\%
        \item \onslide<2->{Want to bridge the gap between theories that operate at different scales.}\\%
        \item \onslide<3->{$\rightarrow$ numerical methods(e.g. lattice QCD)}\\%
    \end{itemize}
\end{columns}
\note{The most fundamental theory we currently have of nuclear physics is QCD. Understanding QCD will help us understand nuclear physics and more \textit{emergent} theories. But to bridge the gaps between these theories is difficult, as QCD contains a large number of degrees of freedom. Thus, a numerical approach is needed.}
\end{frame}

\begin{frame}{We currently have...}
\begin{itemize}[<+->]
    \item A theory for QCD in the \textbf{continuum}.
    \item Which we solve QCD by a Feynman \textbf{path integral}.
    \item We want to solve this \textbf{numerically}.
    \item $\rightarrow$ need to \textbf{discretize} the path integral.
\end{itemize}
\onslide<5->{But first, we need to know \textit{what} a path integral is.}
\end{frame}

\begin{frame}[t]{How we measure: path integrals}
\onslide<1->{Going from $t_0$ at $A$ to a time $t_1$ at $B$ can be given in terms of a path integral.}%
\only<2>{\begin{center}
    \includegraphics[trim={0.0cm 0.5cm 0.0cm 1.0cm},width=0.5\linewidth]{../figures/path-integral-simplified-classic/path-integral-simplified.pdf}
\end{center}}
\only<3->{\begin{center}
    \includegraphics[trim={0.0cm 0.5cm 0.0cm 1.0cm},width=0.5\linewidth]{../figures/path-integral-simplified/path-integral-simplified.pdf}
\end{center}}
\only<2>{Classically, only one possible path obtained from the principle of least action.\\}%
\only<3->{In quantum mechanics, there is lots of possible paths.\\}%
\only<4->{Principle of least action (or stationary condition): $\frac{\delta}{\delta x(t)} \left( S\left[ x(t) \right] \right) = 0$\\}%
\only<5->{\textit{Sum over all possible paths $\rightarrow$ the most likely path.\\}}%
\note{
\begin{itemize}
    \item \onslide<2->{Classically we only have one available path given to us from the principle of least action.}
    \item \onslide<3->{In quantum mechanics, we have several available paths, which are all weighted by some probability amplitude.}
\end{itemize}
}
\end{frame}

\begin{frame}{Path integrals}
\onslide<1->{Given a field $\phi_M$ in Minkowski space, the \textit{partition function $Z$} is given by
\begin{align*}
    Z &= \int \mathcal{D}\phi_M \e^{\frac{i}{\hbar}S_M[\phi_M]} \\
    &\downarrow \quad \hbar=1, \quad \tau \rightarrow -it \qquad \text{imaginary time($\rightarrow$ Euclidean space)!} \\
    &= \int \mathcal{D}\phi \e^{-S[\phi]}
\end{align*}%
where $\mathcal{D}$ is an integration of all possible paths in space.\\}
\onslide<2->{An observable $O$ becomes,
\begin{align*}
    \expect{O} = \frac{1}{Z} \int\mathcal{D}\phi O[\phi] \e^{-S[\phi]}
\end{align*}%
with action given in terms of spacetime integral of the Lagrangian $\mathcal{L}$ \begin{align*}
    S = \int \dd^4 x \mathcal{L}
\end{align*}}%
\onslide<3->{Difficult to calculate the all possible paths $\rightarrow$ discretize spacetime}%
\note{\begin{itemize}
    \item <1->{We go to imaginary time in order to avoid oscillations due to the complex phase.}
    \item <1->{We must also change the action $S$, so it is expressed in Euclidean space.}
    \item <1->{We will \textbf{remain in Euclidean space} for the rest of the presentation.}
    \item <2->{Then, the expectation value given in terms to all possible paths with $Z$ as a weight.}
    \item <3->{It is difficult to calculate all of the possible paths, so instead we discretize spacetime with the field value at every spacetime point instead.}
\end{itemize}
}
\end{frame}

\begin{frame}{Discretizing the path integral}
\begin{center}
    \includegraphics[width=0.8\textwidth]{../figures/lattice-discretized/lattice-discretized.pdf}
\end{center}
\note{$N$ is the number of points on the lattice.}
\end{frame}

\begin{frame}{Discretizing the path integral}
\onslide<1->{Path integral integration measure becomes
\begin{align*}
    \int \mathcal{D}\phi = \prod_{x_\mu} \int \dd \phi_{x_\mu}
\end{align*}}%
\onslide<2->{\noindent We integrate over each spacetime point.\\}%
\onslide<3->{$32 \times 32 \times 32 \times 32 = 2^{20} \rightarrow n^{2^{20}}$ integration points.\\}%
\onslide<4->{A \textbf{statistical approach} using importance sampling is needed for generating gauge configurations.\\}%
% \onslide<5->{An expectation value becomes%
% \begin{align*}
% \expect{O} = \frac{1}{N_\mathrm{cfg}} \sum_{i=1}^{N_\mathrm{cfg}} O[\phi_i] + \mathcal{O}\left(\frac{1}{\sqrt{N_\mathrm{cfg}}}\right)
% \end{align*}
% where $\phi_i$ is generated gauge configuration.}
\note{\begin{itemize}
    \item <1->We now have a integral over the field at every spacetime point.
    \item <3->This poses another challenge, namely that we quickly have a large number of integrals. Say for a $32^4$ lattice, we will have $n^{2^{20}}$ integration points using conventional integration methods, i.e. the trapezoidal method.
    \item <4->This require us to use statistical methods with some sort of importance sampling.
    % \item <5->Since we now are using a statistical method, the challenge is now to generate configurations of the spacetime that lie in the vicinity of the action minimum. The expectation value then becomes the average of such configurations, and the error of these scales with $1/\sqrt{N_\mathrm{cfg}}$.
\end{itemize}}
\end{frame}

\begin{frame}{Configurations: comparing QCD and the Ising model}
\onslide<1->{The 2D Ising model is given by:%
\begin{align*}
    H=-J\sum_{\expect{i,j}} S_i S_j - h\sum_i S_i
\end{align*}}%
\begin{itemize}
    \item \onslide<2->{Ising model only has two possible values at a spin site $S_i$: $\uparrow$, $\downarrow$}
    \item \onslide<3->{QCD many more degrees of freedom: quarks, gluons, color, charge, ...}    
    \item \onslide<4->{With $N=10$, a lattice in the Ising model has the size $10 \times 10 = 100$.}
    \item \onslide<5->{A lattice in LQCD is however $N^4 = 10000$.}
\end{itemize}
\end{frame}

\begin{frame}{Then what is a configuration?}
\onslide<1->{Looking at a spin lattice of the Ising model,}%
\onslide<1->{\begin{columns}
    \column{0.5\textwidth}
    \begin{minipage}[c][0.5\textheight][c]{\linewidth}
    \begin{center}
        \includegraphics[trim={1.5cm 0.5cm 1.5cm 0.5cm},width=0.85\textwidth]{../figures/ising_lattices/ising_t1_1.pdf}
        % \includegraphics[width=0.8\textwidth]{../figures/ising_lattices/ising_t1_1.pdf}
    \end{center}
    \end{minipage}
    \column{0.5\textwidth}
    \begin{minipage}[c][0.5\textheight][c]{\linewidth}
    \begin{center}
        \includegraphics[trim={1.5cm 0.5cm 1.5cm 0.5cm},width=0.85\textwidth]{../figures/ising_lattices/ising_t2_1.pdf}
        % \includegraphics[width=0.8\textwidth]{../figures/ising_lattices/ising_t2_1.pdf}
    \end{center}
    \end{minipage}
\end{columns}}%
\begin{itemize}
    \item \onslide<2->{A \textbf{configuration} in the Ising model is a given \textit{arrangement of the spins}.}
    \item \onslide<3->{A \textbf{configuration} in LQCD is a given \textit{arrangement of the gauge field}.}
\end{itemize}%
\note{\onslide<1->{Two possible values indicated by either black or white dots.}}
\end{frame}

\begin{frame}{Sampling configurations}
\onslide<1->{An expectation value becomes%
\begin{align*}
\expect{O} = \frac{1}{N_\mathrm{cfg}} \sum_{i=1}^{N_\mathrm{cfg}} O[\phi_i] + \mathcal{O}\left(\frac{1}{\sqrt{N_\mathrm{cfg}}}\right)
\end{align*}
where $\phi_i$ is a generated gauge configuration(or just a general configuration).}
\note{\begin{itemize}
    \item <1->Since we now are using a statistical method, the challenge is now to generate configurations of the spacetime that lie in the vicinity of the action minimum. The expectation value then becomes the average of such configurations, and the error of these scales with $1/\sqrt{N_\mathrm{cfg}}$.
\end{itemize}}
\end{frame}

\begin{frame}{QCD on the lattice}
\begin{center}
    \includegraphics[width=0.6\textwidth]{../figures/lattice_quarks_and_gluons.png}
\end{center}
\small{\url{http://www.jicfus.jp/en/wp-content/uploads/2012/12/LatticeQCD.png}}
\note{
\begin{itemize}
    \item The lattice is a cube in 4D.
    \item Quarks at lattice, gluons in-between (\textbf{links}).
    \item Maintains the $\SU(3)$ symmetry by introducing links.
    \item Closed loops are gauge invariant.
    \item Smallest possible object: the plaquette.
    \item Paths of links with fermions as end points are gauge invariant.
    \item However, from now on we will ignore any fermions.
\end{itemize}
}
\end{frame}

\begin{frame}{From QCD to pure \texorpdfstring{$\SU(3)$}{SU3} Yang-Mills}
We exclude fermions to only look at the gauge fields,
\begin{align*}
S_G = \frac{1}{2} \int \dd^4 x \tr \left(G_{\mu\nu}\right)^2
\end{align*}
\end{frame}

\begin{frame}{Links}
\begin{itemize}
    \item \onslide<1->{\textit{Links} $U_\mu(n)$ tell us how the gauge field at lattice location $n$ changes in a given direction $\hat{\mu}$}
    \item \onslide<2->{Four links at every lattice site(one for each Lorentz index). Opposite direction given by its inverse.}
    \item \onslide<3->{Links are complex $3\times 3$ matrices of the group $\SU(3)$ with properties of,}
\end{itemize}%
\onslide<4->{\begin{align*}
    U^\dagger_\mu(x) = U^{-1}_\mu (x), \qquad \det \left(U_\mu(x)\right) = 1.
\end{align*}}%
\onslide<5->{%
\noindent%
From this we can build a lattice action,
\begin{align*}
S_G[U] = \frac{\beta}{3} \sum_{n\in\Lambda} \sum_{\mu<\nu} \Re \tr \left[ 1 - U_\mu(n) U_{\nu}(n+\hat{\mu}) U_{\mu}(n+\hat{\nu})^\dagger U_{\nu} (n)^\dagger \right],
\end{align*}%
with $\beta=6/g_S^2$}%
\note{%
\onslide<5->{%
Notice that in this expression we have that the four links that forms a \textit{plaqeutte}, which is the smallest possible gauge invariant object, are dependent on the neighboring object at $\hat{\mu}$ and $\hat{\nu}$.\\
}
\onslide<6->{Since we are dealing with $\SU(3)$ matrices, we have that we in reality are performing pure matrix algebra. Due to the size of the lattice we would like to split this calculation.}
}
\end{frame}

\begin{frame}{Parallelization: distributing the problem}
\onslide<2->{Number of points in a lattice:
\begin{align*}
    \stackunder{\underbrace{N^3}}{\text{Spatial}} \times \stackunder{\underbrace{N_T}}{\text{Temporal}} \times \stackunder{\underbrace{4}}{\text{Links}} \times \stackunder{\underbrace{9}}{\text{$\SU(3)$ matrix}} \times \stackunder{\underbrace{2}}{\text{$\mathbb{C}$-numbers}} = 72 N^3N_T,
\end{align*}\\}%
\onslide<3->{Too large to solve on any single computer.}%
\note{\onslide<3->{Large number of d.o.f.'s $\rightarrow$ we want to split the calculation.}}
\end{frame}

\begin{frame}{Parallelization: splitting the hypercube}
\only<1>{\begin{center}
    \includegraphics[width=0.7\textwidth]{../figures/parallelization-lattice/parallelization-lattice.pdf}
\end{center}
}
\only<2>{\begin{center}
    \includegraphics[width=0.7\textwidth]{../figures/parallelized-lattice/parallelized-lattice.pdf}
\end{center}
}
\end{frame}

\begin{frame}{Parallelization: shifts}
\onslide<1->{We need a message passing interface for communication(MPI).\\}%
\onslide<2->{Implemented \textit{shifts} for sharing data.\\}%
\onslide<3->{%
\begin{center}
    \includegraphics[width=0.6\textwidth]{../figures/illustrations/implementation/shift/shift.pdf}
\end{center}
}
\end{frame}

\begin{frame}{So far, we have ...}
\begin{itemize}[<+->]
    \item a procedure for calculating the action using links.
    \item a statistical Monte Carlo method for solving the path integral.
    \item We have a method for parallelization for handling the computations.
\end{itemize}
\onslide<4->{However, some observable are problematic...}
\note{
\begin{itemize}
    \item <1->We calculate the action using links.
    \item <2->We will use the Metropolis Monte Carlo method for solving the path integral and generating configurations.
    \item <3->We have a method for handling the computations.
    \item <4->However, some observables are problematic, and we need to apply some method of renormalization in order to retrieve sensible results..
\end{itemize}}
\end{frame}

\begin{frame}{Gradient flow}
\begin{align*}
&\partial_{t_f} B_\mu(x, t_f) = D_{\nu} G_{\nu \mu}(x,t_f) \\
&B_\mu(x, t_f)\big|_{t_f=0} = A_\mu(x)
\end{align*}
\begin{itemize}
    \item \onslide<1->{Solves this by integrating along $t_f$ called \textit{flow time} \footnote{\citet{luscher_properties_2010}}}
    \item \onslide<2->{$B_\mu(x,t_f)$ is the gauge field $A_\mu(x)$ at a flow time $t_f$.}
    \item \onslide<3->{$D_\nu = \partial_{\nu} + [B_\mu(x,t_f), \cdot]$}
    \item \onslide<4->{Field strength tensor: $G_{\mu\nu}(x,t_f) = \partial_\mu B_\nu(x,t_f) - \partial_\nu B_\mu(x,t_f) - i[B_\mu(x,t_f), B_\nu(x,t_f)]$}
\end{itemize}%
\onslide<5->{An analogy: the diffusion equation: 
\begin{align*}
    \frac{\partial}{\partial t_f} B_\mu(x,t_f) \approx \partial^2_x B_\mu(x,t_f)
\end{align*}}
\end{frame}

\begin{frame}{Gradient flow II}
\begin{itemize}
    % \item \onslide<1->{Allows for a scale setting using the energy density: $t_f^2\expect{E}|_{t_f=t_0}=0.3$}
    \item \onslide<1->{The gauge field at $t_f>0$ is a \textbf{smooth, renormalized field}.}
    \item \onslide<2->{Allows us to measure certain quantities such as the \textbf{topological charge}, Q}
\end{itemize}
\note<3->{Solve on the lattice using the symplectic, structure preserving Runge-Kutta 3 solver\footnote{\citet{munthe-kaas_runge-kutta_1998}}}
\end{frame}

\begin{frame}{Gradient flow III: topological charge}
\begin{center}
    \animategraphics[loop,controls,width=0.7\linewidth,scale=1]{10}{../figures/latviz/topc_TE21_lowres/flowQ_TE21-}{00}{60}
\end{center}
\tiny{\textit{Animation created using LatViz.}}
\note{Evolution of topological charge $Q$ in flow time.}
\end{frame}

%%%%%%%%%%%%%%%%%%%%%%%%%%%%%%%%%%%%%%%%%%%%%%%%%%%%%%%%%%%%%%%%%%%%%%%%%%%%%%%%%%%%%%%%%%
%  ____                 _ _
% |  _ \ ___  ___ _   _| | |_ ___
% | |_) / _ \/ __| | | | | __/ __|
% |  _ <  __/\__ \ |_| | | |_\__ \
% |_| \_\___||___/\__,_|_|\__|___/
%%%%%%%%%%%%%%%%%%%%%%%%%%%%%%%%%%%%%%%%%%%%%%%%%%%%%%%%%%%%%%%%%%%%%%%%%%%%%%%%%%%%%%%%%%
\section{Results}

\begin{frame}{Ensembles}
% Include size and time
Points in lattice given by $N^3 \times N_T$.
\begin{table}
    \centering
    \begin{tabular}{l r r r r r r}
        \toprule
        Ensemble & $\beta=6/g_S^2$ & $N$ & $N_T$ & $N_\mathrm{cfg}$ & $a$ $[\fm]$ & Config. size$[$GB$]$ \\ 
        \midrule
        $A$   & 6.0  & 24 & 48 & 1000 & $0.0931(4)$ & 0.356 \\
        $B$   & 6.1  & 28 & 56 & 1000 & $0.0791(3)$ & 0.659 \\
        $C$   & 6.2  & 32 & 64 & 2000 & $0.0679(3)$ & 1.125 \\
        $D_1$ & 6.45 & 32 & 32 & 1000 & $0.0478(3)$ & 0.563 \\
        $D_2$ & 6.45 & 48 & 96 & 250  & $0.0478(3)$ & 5.695 \\
        \bottomrule
    \end{tabular}
\end{table}
\onslide<2->{A scale $t_0$ was set using the energy and gradient flow.}
% \begin{itemize}[<+->]
%     \item We use $N_\mathrm{corr}=1600$ for $\beta=6.45$ ensembles, $N_\mathrm{corr}=600$ for the rest.
%     \item $N_\mathrm{up}=30$.
% \end{itemize}
\note{
\begin{itemize}
    \item I implemented the methods discussed under a code I call GLAC, and will now present some of the results I generated using this code.
    \item The main ensembles made for this thesis.
    \item An ensemble is a collection of configurations using similar size and lattice spacing.
    \item Notice that the size of a lattice is 16 times larger when doubling the dimensions.
    \item \textbf{Ensemble $D_2$ only has 250 configurations}.
    \item Every configuration was flown with $N_\mathrm{flow}=1000$ flow steps.
    \item Since there is are so few data configurations, resampling techniques such as bootstrapping were implemented.
    \item <2->{A scale $t_0$ was set, in which more details exist in the backup slides.}
\end{itemize}
}
\end{frame}

% \begin{frame}{Lattice sizes}
% \begin{table}
%     \centering
%     \begin{tabular}{l r r r}
%         \toprule
%         Ensemble     & $N$             & $L=Na$ $[\fm]$  & $a$ $[\fm]$     \\ \midrule
%         $A$          & $24$            & $2.235(9)$      & $0.0931(4)$     \\ 
%         $B$          & $28$            & $2.214(10)$     & $0.0791(3)$     \\ 
%         $C$          & $32$            & $2.17(1)$       & $0.0679(3)$     \\ 
%         $D_1$        & $32$            & $1.530(9)$      & $0.0478(3)$     \\ 
%         $D_2$        & $48$            & $2.29(1)$       & $0.0478(3)$     \\ 
%         \bottomrule
%     \end{tabular}
% \end{table}
% Charge radius of a proton: $\sim 0.85$ fm.
% \note{The lattice sizes.}
% \end{frame}


% TODO: determine how much detail I should explain topological charge in.
\begin{frame}{Topological charge}
\begin{itemize}
    \item \onslide<1->{QCD vacua(i.e. gauge fields) can be classified by their topological properties.}
    \item \onslide<2->{In this vacua, \textbf{instantons} are local minima of the Yang-Mills action in Euclidean space, as they are solutions to the e.o.m. }
    \item \onslide<3->{\textbf{Topological charge} $Q$ can be viewed as a ``measure'' of instantons.}
\end{itemize}
\onslide<4->{\begin{align*}
    Q = a^4 \sum_{n\in\Lambda} q(n),
\end{align*}%
with the charge density given by
\begin{align*}
    q(n) = \frac{1}{32\pi^2} \epsilon_{\mu\nu\rho\sigma} \tr\left[F_{\mu\nu}(n)F_{\rho\sigma}(n)\right].
\end{align*}}%
\onslide<5->{\noindent%
Integer valued and equally probably to have negative charge as positive,
\begin{align*}
    \expect{Q} = 0
\end{align*}}
\note{\begin{itemize}
    % \item <2->Measuring \textbf{topological charge} is a measure of the \textit{Winding number} of the gauge field.
    \item <2->\textbf{Instantons} can be viewed as local minima to the Yang-Mills action in Euclidean space, as they are localized solutions to the classical finite action equations of motion in Euclidean space.
    \item <3->Instantons are significant in LQCD because of \textit{critical slowing down} which we will return to later.
    \item <4->The charge is a \textbf{sum over the local charge} at every point in the gauge field. 
    \item <4->Can in a very crude manner be viewed as the ``curl'' of the gauge fields due to the Levi-Civita tensor.
    \item <4->The \textbf{expectation value} of \textbf{$Q$ is zero} due to it being \textbf{parity odd}.
    \item <5->The charge is integer valued.
\end{itemize}}
\end{frame}

\begin{frame}{Topological charge distribution}
\vspace{-5.0pt}
\begin{center}
    \includegraphics[trim={0.2cm 0.0cm 0.2cm 1.0cm},clip,scale=0.7]{../../LQCD/LatticeAnalyser/figures/beta61_16x32_run/topc/topc_multihistogram_beta6_1}
\end{center}
\note{Histograms for the $Q$ for ensemble $G$ with a lattice of size $N^3\times N_T=16^3\times 32$ with $\beta=6.1$, taken at different flow times $t_f/a^2=0.0$, $1.0$, $4.0$ fm.}
\end{frame}

\begin{frame}{Topological charge}
\begin{center}
    \animategraphics[loop,controls,width=0.7\linewidth,scale=1]{10}{../figures/latviz/topc_TF500_lowres/flowQ_TF500-}{00}{63}
\end{center}
\tiny{\textit{Animation created using LatViz.}}
\end{frame}

% \begin{frame}{Topological charge distribution in flow time}
% \vspace{-5.0pt}
% \begin{center}
%     \includegraphics[trim={0.2cm 0.0cm 0.2cm 1.0cm},clip,scale=0.7]{../../LQCD/LatticeAnalyser/figures/topc_modes_analysis/topc_modes_tf250.pdf}
% \end{center}
% \note{Histograms of topological charge for the supporting ensembles seen at $t_f/a^2=0.25$ fm.}
% \end{frame}

\begin{frame}{Topological charge for our main ensembles}
\vspace{-5.0pt} 
\begin{center}
    \includegraphics[trim={0.2cm 0.0cm 0.2cm 1.0cm},clip,scale=0.7]{../../LQCD/LatticeAnalyser/figures_b645_32xx4_full/data11/post_analysis/topc/post_analysis_topc_bootstrap.pdf}
\end{center}
\note{\begin{itemize}[<+->]
    \item Topological charge $Q$ as evolved in flow time for the five main ensembles.
    \item The ensemble $D_1$ is okay to include, as the topological charge becomes independent of the volume around a side length of $\sim 1.2$-$1.3$ fermi.
    \item Bootstrapped data with $N_\mathrm{bs}=500$ bootstrap samples.
    \item Corrected for autocorrelations with $\sigma = \sqrt{2 \tau_\mathrm{int}} \sigma_0$.
\end{itemize}
}
\end{frame}

\begin{frame}{Autocorrelations}
\begin{itemize}
    \item \onslide<1->{Why is the charge not centered around zero for certain ensembles?\\}
    \item \onslide<2->{Let us look at the \textbf{autocorrelation} - the measure for correlations between gauge configurations in Monte Carlo time.\\}
    \item \onslide<3->{The autocorrelation is given as $\Gamma(t) = \expect{(x_i - x)(x_{i+t} - x)}$ and $\tau_\mathrm{int} = \frac{1}{2} + \sum^{\infty}_{t=1}\frac{\Gamma(t)}{\Gamma(0)}$.\\}
    \item \onslide<4->{\textbf{Zero autocorrelation} corresponds to $\tau_\mathrm{int} = 0.5$\\}
\end{itemize}
\vfill
\vfill
\vfill
\onslide<5->{\begin{columns}
    \column{0.45\textwidth}
    \begin{minipage}[b][0.5\textheight][b]{\linewidth}
        \begin{figure}
            \centering
            \includegraphics[trim={0.2cm 0.0cm 0.2cm 1.0cm},clip,width=0.95\linewidth]{../../LQCD/LatticeAnalyser/figures/data11/beta62/topc/topc_mchistory_flowt0999_beta6_2.pdf}
            \captionsetup{labelformat=empty}
            \caption{Ensemble $C$, $32^3 \times 64$, $\beta=6.2$}
        \end{figure}
    \end{minipage}
    \column{0.45\textwidth}
    \begin{minipage}[b][0.5\textheight][b]{\linewidth}
        \begin{figure}
            \centering
            \includegraphics[trim={0.2cm 0.0cm 0.2cm 1.0cm},clip,width=0.95\linewidth]{../../LQCD/LatticeAnalyser/figures/data11/beta645/topc/topc_mchistory_flowt0999_beta6_45.pdf}
            \captionsetup{labelformat=empty}
            \caption{Ensemble $D_2$, $48^3 \times 96$, $\beta=6.45$}
        \end{figure}
    \end{minipage}
\end{columns}}
\note{\begin{itemize}
    \item Charge around zero for ensemble $C$, but not as evident for ensemble $D_2$.
    \item Points clearly dependent on the previous point in $D_2$.
\end{itemize}}
\end{frame}

\begin{frame}{Topological charge autocorrelation}
\vspace{-5.0pt}
\begin{center}
    \includegraphics[trim={0.2cm 0.0cm 0.2cm 1.0cm},clip,scale=0.7]{../../LQCD/LatticeAnalyser/figures_b645_32xx4_full/data11/post_analysis/topc/post_analysis_topc_bootstrap_autocorr.pdf}
\end{center}
\note{
\begin{itemize}
    \item The integrated autocorrelation $\tau_\mathrm{int}$ for topological charge for the five main ensembles.
\end{itemize}
}
\end{frame}

\begin{frame}{Critical slowing down}
\begin{itemize}
    \item<1->\onslide<1->{\textbf{Critical slowing down} is the phenomena where we as the lattice spacing $a$ \textit{decreases} the required energy to tunnel from one topological sector to another \textit{increase}. \\}%
    \item<2->\onslide<2->{In the continuum, going from one topological sector, a region with similar topological charge, to another require infinite energy. As $a\rightarrow 0$, the amount of effort required to change the configuration increases.\\}%
    \item<3->\onslide<3->{$\rightarrow$ many more lattice updates are required in order to have independent gauge configurations.}%
\end{itemize}
\note{
\begin{itemize}
    \item <2->That is, going from one instanton sector to another requires many more updates and becomes an inherent problem in all LQCD calculations.
    \item <3->In the continuum it would require an \textbf{infinite amount of energy} to go from \textbf{one instanton sector to another}. Thus as we \textbf{$a$ approaches the continuum}, the amount of \textbf{effort}(number of updates ect.) required to generate \textbf{independent} gauge configurations \textbf{increases}.
\end{itemize}}
\end{frame}

\begin{frame}{Topological susceptibility}
\onslide<1->{%
The \textit{topological susceptibility} is given by%
\begin{align*}
    \chi_\mathrm{top}^{1/4} = \frac{1}{V^{1/4}}\expect{Q^2}^{1/4}
\end{align*}%
with $V$ being the lattice volume and $\expect{Q^2}$ is the second momenta of the charge. \\}%
\onslide<2->{%
The \textit{Witten-Veneziano relation} is given by%
\begin{align*}
    m_{\eta'}^2 = \frac{2N_f}{f^2_\pi}\chi_\mathrm{top}
\end{align*}}%
\onslide<3->{with}%
\begin{block}{}%
\begin{itemize}%
    \item <3->pion decay constant $f_\pi=0.130(5)/\sqrt{2}$ GeV.
    \item <4->$\eta'$ meson mass $m_{\eta'}=0.95778(6)$ GeV.
    \item <5->$N_f$ is the number of flavors(i.e. quark species involved in $\eta'$.).
\end{itemize}
\end{block}%
\onslide<6->{We expect $N_f=3$.}
\note{
\begin{itemize}
    \item <2->R.h.s. is full QCD and l.h.s is from pure gauge theory.
    \item <3->We can use the Witten-Veneziano formula in order to extract an estimate for $N_f$ using the topological susceptibility.
    \item <6->This can help us understand the "quality" of the ensemble, as we would expect to be around $N_f=3$.
\end{itemize}
}
\end{frame}

\begin{frame}{Topological susceptibility}
\begin{center}
    \includegraphics[trim={0.2cm 0.0cm 0.2cm 1.0cm},clip,scale=0.7]{../../LQCD/LatticeAnalyser/figures_b645_32xx4_full/data11/post_analysis/topsus/post_analysis_topsus_bootstrap.pdf}
\end{center}
\note{\begin{itemize}
    \item The topological susceptibility $\chi^{1/4}_{t_f}$ of the \textbf{main ensembles}.
    \item We have a \textbf{UV divergence at zeroth flow time}, hence to need for gradient flow which renormalizes this quantity.
    \item \textbf{Bootstrapped} $N_\mathrm{bs}=500$ times.
    \item \textbf{Corrected for autocorrelations} with $\sigma = \sqrt{2 \tau_\mathrm{int}} \sigma_0$.
\end{itemize}}
\end{frame}

% \begin{frame}{Topological susceptibility continuum extrapolation}
% \begin{table}
%     \centering
%     \begin{tabular}{l r r r}
%         \toprule
%         Ensemble     & $\chi_{t_f}^{1/4}$ [GeV] & $\chi_{t_f}^{1/4}$ [GeV], corrected & $\sqrt{2\tau_\mathrm{int}}$\\
%         \midrule
%         $A$          & $0.1877(23)$ & $0.1877(24)$  & $1.028(46)$  \\
%         $B$          & $0.1880(21)$ & $0.1880(29)$  & $1.346(81)$  \\
%         $C$          & $0.1853(14)$ & $0.1853(24)$  & $1.762(104)$ \\
%         $D_1$        & $0.1971(22)$ & $0.1971(101)$ & $4.523(675)$ \\
%         $D_2$        & $0.1656(33)$ & $0.1656(86)$  & $2.624(441)$ \\
%         \bottomrule
%     \end{tabular}
% \end{table}
% Error corrected for autocorrelations with $\sigma = \sqrt{2 \tau_\mathrm{int}} \sigma_0$. \\
% Values taken at $\sqrt{8t_f}=0.6$ fm.
% \note{
% \begin{itemize}
%     \item Values extracted at a smearing radius of \textbf{hadronic scales}. That is, we have plateaued and have no discretization effects.
%     \item The topological susceptibility for the main ensembles together with the correction factor from the integrated autocorrelation time. The second column have not had its results corrected by $\sqrt{2\tau_\mathrm{int}}$. None of the results have been analyzed with bootstrapping.
% \end{itemize}
% }
% \end{frame}

% \begin{frame}{Topological susceptibility continuum extrapolation}
% \begin{center}
%     \includegraphics[trim={0.2cm 0.0cm 0.2cm 1.0cm},clip,scale=0.7]{../../LQCD/LatticeAnalyser/figures/data11/post_analysis/topsus/post_analysis_extrapmethodbootstrap_topsus_continuum06_bootstrap.pdf}
% \end{center}
% \note{\begin{itemize}
%     \item A continuum extrapolation of the topological susceptibility $\chi^{1/4}_{t_f}$ for the main ensembles excluding the $D_1$ ensemble.
%     \item The points for $\chi^{1/4}_{t_f}$ is taken at $\sqrt{8t_{f,0}}=0.6$ fm. 
% \end{itemize}}
% \end{frame}

\begin{frame}{Topological susceptibility continuum extrapolation}
\begin{table}
    \centering
    \begin{tabular}{l r r r}
        \toprule
        Ensembles               & $\chi_{t_f}^{1/4}\left( \expect{Q^2} \right)$ [GeV]   & $N_f$         & $\chi^2/\mathrm{d.o.f}$ \\ \midrule
        $A$, $B$, $C$, $D_2$    & $0.179(10)$                                           & $3.75(29)$    & $2.38$ \\
        $A$, $B$, $C$, $D_1$    & $0.186(6)$                                            & $3.21(25)$    & $0.83$ \\
        % $B$, $C$, $D_1$         & $0.187(24)$                                           & $3.18(24)$    & $1.63$ \\ 
        % $B$, $C$, $D_2$         & $0.166(24)$                                           & $5.06(39)$    & $2.05$ \\ 
        $A$, $B$, $C$           & $0.184(6)$                                            & $3.37(26)$    & $0.33$ \\
        \bottomrule
    \end{tabular}
\end{table}
\end{frame}

\begin{frame}{The fourth cumulant}
\begin{align*}
    \expect{Q^4}_c = \frac{1}{V^2}\left(\expect{Q^4} - 3\expect{Q^2}^2\right).
\end{align*}%
From this, we can also measure the ratio $R$,
\begin{align*}
    R = \frac{\expect{Q^4}_c}{\frac{1}{V}\expect{Q^2}} = \frac{1}{V}\frac{\expect{Q^4} - 3\expect{Q^2}^2}{\expect{Q^2}},
\end{align*}
\note{
    \begin{itemize}
        \item Highly unstable, as we shall see.
        \item Will provide insight into the goodness of our ensembles.
        \item An $R$-value away from 1 will indicate that QCD cannot be described by the dilute instanton gas model.
    \end{itemize}
}
\end{frame}

\begin{frame}{The fourth cumulant}
% Plot goes here
\begin{center}
    \includegraphics[trim={0.2cm 0.0cm 0.2cm 1.0cm},clip,scale=0.7]{../../LQCD/LatticeAnalyser/figures_b645_32xx4_full/data11/post_analysis/topcr/post_analysis_topcr_bootstrap.pdf}
\end{center}
\note{\begin{itemize}
    \item The fourth cumulant ratio $R=\expect{Q^4}_C/\expect{Q^2}$.
    \item The results was analyzed using $N_\mathrm{bs}=500$ bootstrap samples, with the error corrected for by $\sqrt{2\tau_\mathrm{int}}$.
\end{itemize}}
\end{frame}

\begin{frame}{The fourth cumulant at reference flow times}
% Tables
\begin{table}
    \scalebox{0.85}{
    \centering
    \begin{tabular}{l r r r r r r}
        \toprule
        Ensemble & $L/a$ & $t_0/a^2$ & $\langle Q^2 \rangle$ & $\langle Q^4 \rangle$ & $\langle Q^4 \rangle_C$ & $R$             \\
        \midrule
        $A$      & $2.24$ & $3.20(3)$ & $0.78(4)$             & $2.13(27)$            & $0.282(67)$            & $0.359(65)$     \\ 
        $B$      & $2.21$ & $4.43(4)$ & $0.81(5)$             & $1.98(23)$            & $0.036(11)$            & $0.044(11)$     \\ 
        $C$      & $2.17$ & $6.01(6)$ & $0.77(4)$             & $1.6(2)$              & $-0.174(40)$           & $-0.226(64)$    \\ 
        $D_1$    & $1.53$ & $12.2(1)$ & $1.00(20)$            & $3.01(1.07)$          & $0.03(12)$             & $0.03(12)$      \\ 
        $D_2$    & $2.29$ & $12.2(1)$ & $0.497(100)$          & $0.64(20)$            & $-0.103(95)$           & $-0.21(23)$     \\ 
        \bottomrule
    \end{tabular}}
\end{table}
\note{The fourth cumulant is taken at their individual reference scales seen in the third column. The data were analyzed with using a bootstrap analysis of $N_{bs}=500$ samples, with error corrected by the integrated autocorrelation, $\sqrt{2\tau_\mathrm{int}}$.}
\end{frame}

\begin{frame}{Comparing fourth cumulant}
\only<1>{We can compare with article by \citet{ce_non-gaussianities_2015}}
\only<2>{\begin{table}
    \centering
    \scalebox{0.85}{
    \begin{tabular}{l r r r r r r r}
        \toprule
        Ensemble        & $\beta$         & $L/a$           & $L$ $[\fm]$     & $a$ $[\fm]$     & $t_0/{a^2}$     & $t_0/{r_0^2}$ & $N_\mathrm{cfg}$  \\
        \midrule
        $F_1$           & $5.96$          & $16$            & $1.632$         & $0.102$         & $2.7887(2)$     & $0.1113(9)$   & 1 440 000 \\
        \addlinespace
        $B_2$           & $6.05$          & $14$            & $1.218$         & $0.087$         & $3.7960(12)$    & $0.1114(9)$   & 144 000 \\ 
        $\tilde{D}_2$   &                 & $17$            & $1.479$         &                 & $3.7825(8)$     & $0.1110(9)$   &         \\
        \addlinespace
        $B_3$           & $6.13$          & $16$            & $1.232$         & $0.077$         & $4.8855(15)$    & $0.1113(10)$  & 144 000 \\ 
        $\tilde{D}_3$   &                 & $19$            & $1.463$         &                 & $4.8722(11)$    & $0.1110(10)$  &         \\
        \addlinespace
        $B_4$           & $6.21$          & $18$            & $1.224$         & $0.068$         & $6.2191(20)$    & $0.1115(11)$  & 144 000 \\ 
        $\tilde{D}_4$   &                 & $21$            & $1.428$         &                 & $6.1957(14)$    & $0.1111(11)$  &         \\ 
        \bottomrule
    \end{tabular}}
    \label{tab:topcr-article-parameters}
\end{table}}
% \only<3>{\begin{table}
%     \centering
%     \scalebox{0.85}{
%     \begin{tabular}{l r r r r} % Quantity Flow-time Article-Value My-Value Article-Me-Ratio
%         \toprule
%         Ensemble        & $\langle Q^2 \rangle_\text{normed}$ & $\langle Q^4 \rangle_\text{normed}$ & $\langle Q^4 \rangle_{C,\text{normed}}$ & $R_\text{normed}$ \\
%         \midrule
%         $F_1$           & $0.728(1)$      & $1.608(4)$      & $0.016(1)$      & $0.022(1)$      \\
%         \addlinespace
%         $B_2$           & $0.772(3)$      & $1.873(19)$     & $0.085(4)$      & $0.110(5)$      \\ 
%         $\tilde{D}_2$   & $0.770(3)$      & $1.817(17)$     & $0.037(4)$      & $0.048(5)$      \\
%         \addlinespace
%         $B_3$           & $0.760(3)$      & $1.805(17)$     & $0.074(3)$      & $0.097(4)$      \\ 
%         $\tilde{D}_3$   & $0.769(3)$      & $1.801(14)$     & $0.027(1)$      & $0.035(1)$      \\
%         \addlinespace
%         $B_4$           & $0.776(3)$      & $1.874(18)$     & $0.069(3)$      & $0.089(4)$      \\ 
%         $\tilde{D}_4$   & $0.785(3)$      & $1.891(17)$     & $0.040(4)$      & $0.052(5)$      \\ 
%         \bottomrule
%     \end{tabular}}
% \end{table}}
\only<3->{\begin{table}
    \centering
    \scalebox{0.85}{
    \begin{tabular}{l l r r r r} % Quantity Flow-time Article-Value My-Value Article-Me-Ratio
        \toprule
        Article          & Thesis & $\text{Ratio}(\langle Q^2 \rangle)$ & $\text{Ratio}(\langle Q^4 \rangle)$ & $\text{Ratio}(\langle Q^4 \rangle_C)$ & $\text{Ratio}(R)$ \\
        \midrule
        $F_1$           & $A$             & $1.08(6)$       & $1.34(18)$      & $19.03(5.81)$   & $17.64(4.48)$   \\ 
        \addlinespace
        $B_2$           & $A$             & $1.02(5)$       & $1.15(15)$      & $3.60(1.09)$    & $3.54(90)$      \\ 
                        & $B$             & $1.04(6)$       & $1.06(11)$      & $0.480(74)$     & $0.46(4)$       \\ 
        \addlinespace
        $\tilde{D}_2$   & $A$             & $1.02(5)$       & $1.19(15)$      & $8.31(1.99)$    & $8.15(1.56)$    \\ 
                        & $B$             & $1.05(6)$       & $1.10(12)$      & $1.1(1)$        & $1.06(3)$       \\ 
        \addlinespace
        $B_3$           & $B$             & $1.06(6)$       & $1.10(12)$      & $0.550(86)$     & $0.52(5)$       \\ 
        \addlinespace
        $\tilde{D}_3$   & $B$             & $1.05(6)$       & $1.11(12)$      & $1.51(23)$      & $1.4(1)$        \\ 
        \addlinespace
        $B_4$           & $C$             & $0.99(5)$       & $0.86(8)$       & $-2.32(46)$     & $-2.35(59)$     \\ 
        \addlinespace
        $\tilde{D}_4$   & $C$             & $0.98(5)$       & $0.85(8)$       & $-3.95(96)$     & $-4.05(1.19)$   \\
        \bottomrule
    \end{tabular}}
    \label{tab:topcr-comparison}
\end{table}}
\note{\begin{itemize}
    \item <2>Parameters of the ensembles presented by \citet{ce_non-gaussianities_2015}. The first column is the ensemble name from the article. The letter indicates the volume, while the subindex indicates the $\beta$ value. Ensembles of similar letters keep approximately the same length $L$.
    % \item <3>Results as presented by \citet{ce_non-gaussianities_2015}, \textbf{normalized by the lattice volume}.
    \item <3>A comparison between the results obtained in this thesis on the fourth cumulant, and by those similar in volume form \citet{ce_non-gaussianities_2015}. \textit{Ratio} indicates that we are dividing our results by the ones in previous table. \textbf{$1$ is perfect overlap}.
\end{itemize}}
\end{frame}

\begin{frame}{The topological charge correlator and the effective glueball mass} 
\onslide<1->{The \textbf{topological charge correlator}%
\begin{align*}
    C(n_t) = \expect{q(n_t)q(0)},
\end{align*}%
is the correlator between two topological charge densities in Euclidean time.\\}
\onslide<2->{The ground state in the correlator is given as%
\begin{align*}
    C(n_t) = A_0 \e^{-n_t E_0} + A_1 \e^{-n_t E_1} + \dots
\end{align*}}%
\onslide<3->{from which the \textbf{effective glueball mass} can be extracted as%
\begin{align*}
    a m_\mathrm{eff} = \log \left(\frac{C(n_t)}{C(n_t + 1)}\right),
\end{align*}}%
\note{\begin{itemize}
    \item In pure Yang-Mills gauge theory, the states are stable.
    \item We will be looking at the state involving topological charge.
\end{itemize}}
\end{frame}

\begin{frame}{The topological charge correlator}
% Two figures for main ensembles
\only<1-2>{\begin{center}
    \includegraphics[trim={0.2cm 0.0cm 0.2cm 1.0cm},clip,scale=0.7]{../../LQCD/LatticeAnalyser/figures/data11/post_analysis/qtq0e/slices/te0000/post_analysis_qtq0e_bootstrap_time_series_tf0_6000.pdf}
\end{center}}%
\only<3->{\begin{center}
    \includegraphics[trim={0.2cm 0.0cm 0.2cm 1.0cm},clip,scale=0.7]{../../LQCD/LatticeAnalyser/figures/data11/beta645-32xx4/qtq0e/tflow0.6000/te0000/qtq0e_bootstrap_Nbs500_beta6_45.pdf}
\end{center}}%
\note{\begin{itemize}
    \item<1-> The topological charge correlator for all of the ensembles except $D_1$. The $x$-axis contains the sink-source separation, as the source $q(0)$ is placed at $t_e=0$ fm, and the sink $q(t_e)$ is taken at $t_e$.
    \item<2-> We since the ensembles are of different lattice sizes, we plot th $D_1$ separately.
    \item<3-> The topological charge correlator for the $D_1$. The source $q(0)$ is placed at $t_e=0$ fm and the sink $q(t_e)$ is taken at $t_e$.
    \item<4-> We would expect an \textbf{exponential dampening}.
\end{itemize}}
\end{frame}

% \begin{frame}{The effective glueball mass}
% A glueball is a bound state of gluons.\\%
% \onslide<2->{The ground state in the correlator is given as%
% \begin{align*}
%     C(n_t) = A_0 \e^{-n_t E_0} + A_1 \e^{-n_t E_1} + \dots
% \end{align*}}%
% \onslide<3->{which can be extracted as%
% \begin{align*}
%     a m_\mathrm{eff} = \log \left(\frac{C(n_t)}{C(n_t + 1)}\right),
% \end{align*}}%
% \note{
% In pure Yang-Mills gauge theory, the states are stable.\\
% We will be looking at the state involving topological charge.
% }
% \end{frame}

\begin{frame}{The effective glueball mass}
\vspace{-0.5cm}
% \only<1>{\begin{center} % TODO: REMOVE - only show at four different flow times
%     \includegraphics[trim={0.2cm 0.0cm 0.2cm 1.0cm},clip,scale=0.7]{../../LQCD/LatticeAnalyser/figures_b645_32xx4_full/data11/post_analysis/qtq0eff/slices/post_analysis_qtq0eff_bootstrap_tf0_6_ma_overlay.pdf}
% \end{center}}
\only<1->{\begin{center}
    \includegraphics[trim={0.2cm 0.0cm 0.2cm 1.0cm},clip,scale=0.7]{../../LQCD/LatticeAnalyser/figures_b645_32xx4_full/data11/post_analysis/qtq0eff/beta6.2-6.0-6.1-6.45-6.45_N1246/post_analysis_qtq0eff_bootstrap_ma.pdf}
\end{center}}
% \only<3->{The effective mass of the glueball seen at four different flow times, $\sqrt{8t_{f,0}}\in[0.1,0.2,0.3,0.4,0.6]$.}
\note{\begin{itemize}[<+->]
    \item The effective mass of the glueball, as extracted from the topological charge correlator in Euclidean time.
    \item Low statistics and critical slowing down $\rightarrow$ poor signal.
\end{itemize}}
\end{frame}

%%%%%%%%%%%%%%%%%%%%%%%%%%%%%%%%%%%%%%%%%%%%%%%%%%%%%%%%%%%%%%%%%%%%%%%%%%%%%%%%%%%%%%%%%%
%   ____                 _           _
%  / ___|___  _ __   ___| |_   _ ___(_) ___  _ __
% | |   / _ \| '_ \ / __| | | | / __| |/ _ \| '_ \
% | |__| (_) | | | | (__| | |_| \__ \ | (_) | | | |
%  \____\___/|_| |_|\___|_|\__,_|___/_|\___/|_| |_|
%%%%%%%%%%%%%%%%%%%%%%%%%%%%%%%%%%%%%%%%%%%%%%%%%%%%%%%%%%%%%%%%%%%%%%%%%%%%%%%%%%%%%%%%%%
\section{Conclusion, future developments and final thoughts}
\begin{frame}{Conclusion}
\begin{itemize}
    \item <1->Created a code (GLAC) capable of generating and flowing gauge configurations.
    % \item <1->Scaling and parameter optimizations % SCALING RUNS, N_UP COULD BE INCREASED, COMPARABLE TO OTHER CODES 
    \begin{itemize}
        \item <2->Verified to match other code bases down to machine precision.
        % \item <2->Room for an increased number of processors.
        % \item <3->$N_\mathrm{up} > N_\mathrm{corr}$
        % \item <4->Machine precision accuracy when comparing with Chroma.
    \end{itemize}
    \item <3->Created a code (LatViz) for visualizing gauge fields.
    % \item <3->$t_0$ and $w_0$ match other papers, e.g. \citet{luscher_properties_2010} and \citet{ce_non-gaussianities_2015}.% REFERENCE SCALE T0 - close to what is found by Luscher, REFERENCE SCALE W0 - close to what is found in paper
    \item <4->$\expect{Q} \neq 0$ for some ensembles. % CHARGE - not zero!
    \item <5->The topological susceptibility $\expect{\chi^{1/4}_f}$ and $N_f$ % TOPSUS AND N_F - compare ts-bootstrap and bootstrap. Re-iterate main results.
    \item <6->$\expect{Q^4}_C$ and $R$. Sensitive quantities - need large statistics.% 4th CUMULANT - quite off. Larger beta than article. Perhaps just poor statistics.
    \item <7->Topological charge correlator $\expect{q(n_t)q(0)}$ and glueball mass. % CORRELATOR AND GLUEBALL MASS
    \item <8->Statistics, autocorrelation and critical slowing down.% STATISTICS AND CRITICAL SLOWing down
\end{itemize}
\note{\begin{itemize}
    \item <1->Checked scaling and parameter optimization.
    % \item <2->We checked \textbf{strong}, \textbf{weak} and \textbf{speedup}, where we appeared to have a plateauing around 512 processors but with room for optimization.
    % \item <3->$N_\mathrm{up}$ could be increased, as it has a smaller impact than $N_\mathrm{corr}$.
    \item <2->Same results as Chroma down to machine precision.
    % \item <4->We also checked the $\epsilon_\mathrm{rnd}$ for matrix generation parameter that it minimized the autocorrelation, and the integration step $\epsilon_f$ .
    % \item <3->$t_0$ and $w_0$ match other papers.
    \item <4->$\expect{Q}$
    \item <5->$\chi^{1/4}_f$. Matches well with other papers
    \item <6->$\expect{Q^4}_C$ and $R$. Sensitive quantity, matches well of first and second moment.
    \item <7->Topological charge correlator $\expect{q(n_t)q(0)}$ and glueball mass.
    \item <8->Critical slowing down, which makes transitioning to a new, independent configuration difficult. This inhibits the gathering of statistics, and helps us explain why we for larger $\beta$ values have fewer independent gauge configurations.
\end{itemize}
}
\end{frame}

\begin{frame}{Future developments and final thoughts}
\begin{itemize}[<+->]
    \item Better statistics - more gauge configurations.
    % \item Better utilization of support ensembles $E$, $F$ and $G$.
    % \item Improve autocorrelation with more updates per single link(increase $N_\mathrm{up}$).
    \item Implement better actions with operators that have smaller error contributions.
    \item Fermions and HMC(Hybrid Monte Carlo).
\end{itemize}
\end{frame}

\begin{frame}
\begin{center}
Thank you for listening.
\end{center}
\vspace{5pt}
\begin{center}
Questions?
\end{center}
\end{frame}

\begin{frame}
\tiny
\bibliography{bibliography/lib.bib}
\end{frame}

%%%%%%%%%%%%%%%%%%%%%%%%%%%%%%%%%%%%%%%%%%%%%%%%%%%%%%
%  _____      _                   _ _     _
% | ____|_  _| |_ _ __ __ _   ___| (_) __| | ___  ___
% |  _| \ \/ / __| '__/ _` | / __| | |/ _` |/ _ \/ __|
% | |___ >  <| |_| | | (_| | \__ \ | | (_| |  __/\__ \
% |_____/_/\_\\__|_|  \__,_| |___/_|_|\__,_|\___||___/
%%%%%%%%%%%%%%%%%%%%%%%%%%%%%%%%%%%%%%%%%%%%%%%%%%%%%%
\section{Extra slides}

%%%%%%%%%%%%%%%%%%%%%%%%%%%%%%%%%%%%%%%%%%%%%%%%%%%%%%%%%%%%%%%%%%%%%%%%%%%%%%%%%%%%%%%%%%%%%%%%%%%%%%%%%%%%%%%%%%%%%%%%%%%%%%%%%%%%%%%%%%%%%%%%%%%%%%%%%%%%%%%%%%%%%%%%%
%  ____            _ _                             _   _           _          _   _                               _                  _  __ _           _   _
% / ___|  ___ __ _| (_)_ __   __ _      ___  _ __ | |_(_)_ __ ___ (_)______ _| |_(_) ___  _ __     __ _ _ __   __| | __   _____ _ __(_)/ _(_) ___ __ _| |_(_) ___  _ __
% \___ \ / __/ _` | | | '_ \ / _` |    / _ \| '_ \| __| | '_ ` _ \| |_  / _` | __| |/ _ \| '_ \   / _` | '_ \ / _` | \ \ / / _ \ '__| | |_| |/ __/ _` | __| |/ _ \| '_ \
%  ___) | (_| (_| | | | | | | (_| |_  | (_) | |_) | |_| | | | | | | |/ / (_| | |_| | (_) | | | | | (_| | | | | (_| |  \ V /  __/ |  | |  _| | (_| (_| | |_| | (_) | | | |
% |____/ \___\__,_|_|_|_| |_|\__, ( )  \___/| .__/ \__|_|_| |_| |_|_/___\__,_|\__|_|\___/|_| |_|  \__,_|_| |_|\__,_|   \_/ \___|_|  |_|_| |_|\___\__,_|\__|_|\___/|_| |_|
%                            |___/|/        |_|
%%%%%%%%%%%%%%%%%%%%%%%%%%%%%%%%%%%%%%%%%%%%%%%%%%%%%%%%%%%%%%%%%%%%%%%%%%%%%%%%%%%%%%%%%%%%%%%%%%%%%%%%%%%%%%%%%%%%%%%%%%%%%%%%%%%%%%%%%%%%%%%%%%%%%%%%%%%%%%%%%%%%%%%%%
\subsection{Scaling, optimization and verification}
\begin{frame}{Scaling}
We checked three types of scaling,
\begin{block}{}
\begin{itemize}
    \item <1->{\textbf{Strong scaling:} \textit{fixed problem} and a \textit{variable} $N_p$ \textit{cores}}
    \item <2->{\textbf{Weak scaling:} \textit{fixed problem per processor} and a \textit{variable} $N_p$ \textit{cores}.}
    \item <3->{\textbf{Speedup:} defined as $S(p) = \frac{t_{N_{p}}}{t_{N_p,0}}$.}
\end{itemize}
\end{block}
\only<1>{\vspace{-10.0pt}\begin{center}\includegraphics[trim={0.4cm 0.0cm 0.4cm 1.2cm},clip,scale=0.35]{../../LQCD/LatticeAnalyser/figures/scaling/strong/strong_all}\end{center}}
\only<2>{\vspace{-10.0pt}\begin{center}\includegraphics[trim={0.4cm 0.0cm 0.4cm 1.2cm},clip,scale=0.35]{../../LQCD/LatticeAnalyser/figures/scaling/weak/weak_all}\end{center}}
\only<3>{\vspace{-10.0pt}\begin{center}\includegraphics[trim={0.4cm 0.0cm 0.4cm 1.2cm},clip,scale=0.35]{../../LQCD/LatticeAnalyser/figures/scaling/strong/speedup_strong_all}\end{center}}
\only<5>{We appear to have a plateau around 512 cores.}
\note{\begin{itemize}
    \item <1->Strong scaling
    \item <2->Weak scaling
    \item <3->The speedup of the configuration generation, flowing, and IO. The speedup is calculated by dividing the run time of each $N_p$ run, with the run time of the run with the least number of processors, $N_p=8$.
    \item <4->The IO was optimized later to be a factor of ten or more faster.
\end{itemize}}
\end{frame}

\begin{frame}{Optimizing the gauge configuration generation}
\only<1>{Generated 200 configurations for a lattice of size $N^3\times N_T = 16^3 \times 32$ and $\beta=6.0$, for combinations of $N_\mathrm{corr}\in[200,400,600]$ and $N_\mathrm{up}\in[10,20,30]$.}
\only<2>{\begin{center}\includegraphics[trim={0.4cm 0.0cm 0.4cm 0.0cm},clip,scale=0.6]{../figures/lattice-updates/topc_autocorr}\end{center}}
\only<3->{\begin{center}\includegraphics[trim={0.4cm 0.0cm 0.4cm 0.0cm},clip,scale=0.6]{../figures/lattice-updates/topc_total_runtime}\end{center}}
\note{
\begin{itemize}
    \item <1->{We run for different values for $N_\mathrm{up}$ and $N_\mathrm{corr}$ to see what gives optimizes \textbf{computational cost} and \textbf{autocorrelation}.}
    \item <1->{The integrated autocorrelation time for topological charge $\expect{Q}$ for a lattice of size $N=16$ and $N_T=32$ with $\beta=6.0$ for combinations of $N_\mathrm{corr}\in[200,400,600]$ and $N_\mathrm{up}\in[10,20,30]$, plotted against flow time $\sqrt{8t_f}$.}
    \item <2->{The time taking to generate 200 configurations and flowing them $N_\mathrm{flow}=250$ flow steps for a lattice of size $N=16$ and $N_T=32$, with $\beta=6.0$ for combinations of $N_\mathrm{corr}\in[200,400,600]$ and $N_\mathrm{up}\in[10,20,30]$.}
    \item <3->{What we see is that increasing $N_\mathrm{up}$ is a cheaper alternative compared to using $N_\mathrm{corr}$}
\end{itemize}}
\end{frame}

\begin{frame}{Verifying the code}
\begin{itemize}[<+->]
    \item \textbf{Unit testing.} $\SU(3)$, $\SU(2)$ multiplications.
    \item \textbf{Integration testing.} Random matrix generation, lattice objects, parallelization, ect.
    \item \textbf{Validation testing.} Cross checking results with a configuration from Chroma.
\end{itemize}
\end{frame}

\begin{frame}{Verifying the integration}
\only<1>{Testing the integrator for different integration steps $\epsilon_f$.
\begin{table}
    \centering
    \resizebox{0.8\columnwidth}{!}{%
    \begin{tabular}{c c c c c c c c c c c}
    \toprule
    $\epsilon_f$ & $0.001$ & $0.005$ & $0.007$ & $0.009$ & $0.01$ & $0.02$ & $0.03$ & $0.05$ & $0.1$ & $0.5$ \\ \bottomrule
    \end{tabular}
    }
\end{table}}
\only<2>{Lattice size $N^3\times N_T = 24^3 \times 48$ with $\beta=6.0$.
\vspace{-10.0pt}
\begin{center}
    \includegraphics[trim={0.4cm 0.0cm 0.4cm 0.0cm},clip,scale=0.6]{../figures/flow-epsilon/energy}
\end{center}}
\only<3->{The absolute difference between the smallest flow time $\epsilon_f=0.001$ and those shown previously.
\vspace{-10.0pt}
\begin{center}
    \includegraphics[trim={0.4cm 0.0cm 0.4cm 0.0cm},clip,scale=0.6]{../figures/flow-epsilon/energy_diff_absolute}
\end{center}}
\note{
\begin{itemize}
    \item<1-> The values we will test the integrator against.
    \item<2-> The energy flowed for different the different $\epsilon_f$ values.
    \item<3-> The absolute difference between the smallest flow time $\epsilon_f=0.001$ and those listed in previous table.
    \item<3-> The reason for \textbf{only having two points} is due to the fact that we are only \textbf{comparing points} that are \textbf{close to each other in flow time}. If we were to have more points, we would have to double the number of flow time steps for the smallest lattices.
    \item<4-> An \textbf{example} of the flowing, can be seen by observing the \textbf{energy evolving over flow time}.
\end{itemize}}
\end{frame}


%%%%%%%%%%%%%%%%%%%%%%%%%%%%%%%%%%%%%%%%%%%%%%%%%%%%%%%%%%%%%%%%%%%%%%%%%%%%%%%%%%%%%%%%%%
%   ___   ____ ____
%  / _ \ / ___|  _ \
% | | | | |   | | | |
% | |_| | |___| |_| |
%  \__\_\\____|____/
%%%%%%%%%%%%%%%%%%%%%%%%%%%%%%%%%%%%%%%%%%%%%%%%%%%%%%%%%%%%%%%%%%%%%%%%%%%%%%%%%%%%%%%%%%
\subsection{QCD Extras}
\begin{frame}{The non-linearity of QCD}
The QCD Lagrangian
\begin{align*}
    \mathcal{L}_\mathrm{QCD} = \sum^{N_f}_{f=1} \bar{\psi}^{(f)} \left(i\slashed{D} - m^{(f)}\right) \psi^{(f)} - \frac{1}{4}G^a_{\mu\nu}G^{a\mu\nu},
\end{align*}
with action
\begin{align}
    S = \int \dd^4 x \mathcal{L}_\mathrm{QCD},
\end{align}
is invariant under a $\SU(3)$ symmetry.
\begin{align*}
    \includegraphics[scale=0.9]{../figures/feynman-diagrams/fermion-gluon-vertex/fermion-gluon-vertex.pdf} \quad
    \includegraphics[scale=0.9]{../figures/feynman-diagrams/three-gluon-vertex/three-gluon-vertex.pdf} \quad
    \includegraphics[scale=0.9]{../figures/feynman-diagrams/four-gluon-vertex/four-gluon-vertex.pdf}
\end{align*}
\note{
    \begin{itemize}
        \item \textit{Gluon self-interaction.}
        \item This central aspect is mostly covered in the pure-gauge/Yang-Mills section of the theory.
        \item \textbf{Two important features}: \textit{confinement} and \textit{asymptotic freedom}.
    \end{itemize}
}
\end{frame}

\begin{frame}{Asymptotic freedom}
    \begin{center}
        \includegraphics[width=0.8\textwidth]{../figures/pdg_asymptotic_freedom-eps-converted-to.pdf}
    \end{center}
    \note{
        \begin{itemize}
            \item The coupling constant \textbf{decreases} as we \textbf{increase} the energy.
            \item Also serves as an \textit{experimental proof} of QCD.
            \item Other lines of \textit{evidence}: triple $\gamma$ decay and muon cross section ratio $R$.
            \begin{itemize}
                \item Triple $\gamma$ decay: the number of colors is included in the cross section, which can be measured experimentally.
                \item Muon cross section ratio $R$: the ratio is dependent on having three colors.
            \end{itemize}
        \end{itemize}
    }
\end{frame}

\subsection{Gradient flow extras}
\begin{frame}{Solving gradient flow with Runge-Kutta 3}
\onslide<1->{With
\begin{align*}
    \dot{V}_{t_f} = - g_S^2 \left\{\partial_{x,\mu} S_G[V_{t_f}] \right\} V_{t_f} = Z(V_{t_f}) V_{t_f},
\end{align*}}
\onslide<2->{and $Z_i=\epsilon_fZ(W_i)$ we get
\begin{align*}
    \begin{split}
        &W_0 = V_{t_f}, \\
        &W_1 = \exp\left[\frac{1}{4}Z_0\right]W_0, \\
        &W_2 = \exp\left[\frac{8}{9}Z_1 - \frac{17}{36}Z_0\right]W_1, \\
        &V_{t_f + \epsilon_f} = \exp\left[\frac{3}{4}Z_2 - \frac{8}{9}Z_1 + \frac{17}{36}Z_0\right]W_2,
    \end{split}
\end{align*}
with coefficients from \citet{luscher_properties_2010}.}
\note{
\begin{itemize}
    \item <1->{We rewrite the equations slightly,}
    \item <2->{and use a structure preserving integrator with coefficients from \citet{luscher_properties_2010}.}
    \item <3->{We control the accuracy of this integrator by $\epsilon_f$.}
\end{itemize}}
\end{frame}

%%%%%%%%%%%%%%%%%%%%%%%%%%%%%%%%%%%%%%%%%%%%%%%%%%%%%%%%%%%%%%%%%%%%%%%%%%%%%%%%%%
%     _       _     _ _ _   _                   _                       _ _
%    / \   __| | __| (_) |_(_) ___  _ __   __ _| |  _ __ ___  ___ _   _| | |_ ___
%   / _ \ / _` |/ _` | | __| |/ _ \| '_ \ / _` | | | '__/ _ \/ __| | | | | __/ __|
%  / ___ \ (_| | (_| | | |_| | (_) | | | | (_| | | | | |  __/\__ \ |_| | | |_\__ \
% /_/   \_\__,_|\__,_|_|\__|_|\___/|_| |_|\__,_|_| |_|  \___||___/\__,_|_|\__|___/
%%%%%%%%%%%%%%%%%%%%%%%%%%%%%%%%%%%%%%%%%%%%%%%%%%%%%%%%%%%%%%%%%%%%%%%%%%%%%%%%%%
\subsection{Additional results}
\begin{frame}{Additional ensembles}
\begin{table}
    \centering
    \resizebox{0.8\columnwidth}{!}{%
    \begin{tabular}{l r r r r r r r}
        \toprule
        Ensemble & $N$ & $N_T$ & $N_\mathrm{cfg}$ & $N_\mathrm{corr}$ & $N_\mathrm{up}$ & $a$ $[\fm]$ & $L$ $[\fm]$ \\ 
        \midrule
        $E$ & 8 & 16 & 8135 & 600 & 30 & 0.0931(4) & 0.745(3) \\
        $F$ & 12 & 24 & 1341 & 200 & 20 & 0.0931(4) & 1.118(5) \\
        $G$ & 16 & 32 & 2000 & 400 & 20 & 0.0790(3) & 1.265(6) \\
        \bottomrule
    \end{tabular}
    }
\end{table}
\note{\begin{itemize}
    \item Additional ensembles made in order to illuminate additional aspects of the topological charge.
    \item Supporting ensembles made on Smaug. All ensembles were flown $N_\mathrm{flow}=1000$ steps with $\epsilon_\mathrm{flow}=0.01$.
\end{itemize}}
\end{frame}


\subsubsection{Scale setting}
\begin{frame}{Energy definition}
\onslide<1->{We use $t_0$
\begin{align*}
    E = \frac{a^4}{2|\Lambda|} \sum_{n\in\Lambda}\sum_{\mu,\nu} \left(F^\mathrm{clov}_{\mu\nu}(n)\right)^2 \\
\end{align*}}
\onslide<2->{$F_{\mu\nu}^\mathrm{clov}(n)$ is given by \begin{center}
    \includegraphics[width=0.5\textwidth]{../figures/illustrations/lqcd/clover/clover}
\end{center}}
\note{\begin{itemize}
    \item \onslide<1->{Defined as the field strength tensor squared averaged over all lattice points and directions.}
    \item \onslide<2->{We will use the clover field strength definition in gauge observables.}
    \item \onslide<2->{\textbf{Symmetries} will allow us to \textbf{reduce} the effective \textbf{number of clovers} need to \textbf{calculate from 24 to 6}.}
\end{itemize}}
\end{frame}

\begin{frame}{Energy}
Using scale definition $t_0$ from \citet{luscher_properties_2010},
\begin{align*}
    \left\{t^2_f \expect{E(t)} \right\}_{t_f = t_{0}} = 0.3.
\end{align*}
\begin{center}
    \includegraphics[trim={0.2cm 0.0cm 0.2cm 0.5cm},clip,width=0.5\textwidth]{../../LQCD/LatticeAnalyser/figures_b645_32xx4_full/data11/post_analysis/energy/post_analysis_energy_bootstrap_zoomed.pdf}
\end{center}
\end{frame}

\begin{frame}{Scale setting \texorpdfstring{$t_0$}{t0}}
\only<1>{
\begin{center}
    \includegraphics[trim={0.2cm 0.25cm 0.2cm 1.2cm},clip,width=0.8\textwidth]{../../LQCD/LatticeAnalyser/figures/data11/post_analysis/energy/post_analysis_extrapmethodbootstrap_t0reference_continuum_bootstrap.pdf}
\end{center}
Continuum extrapolation using ensembles $A$, $B$, $C$, and $D_2$ gives $t_{0,\mathrm{cont}}/r_0^2 = 0.11087(50)$.}
\only<2->{This matches the values retrieved by \citet{luscher_properties_2010},
\begin{center}
    \includegraphics[width=0.5\textwidth]{../figures/luscher_extrap_t0.png}
\end{center}}
\note{
\begin{itemize}
    \item \only<1->{The continuum extrapolation $a\rightarrow 0$ for $t_0$ of the four ensembles $A$, $B$, $C$, and $D_2$.}
    \item \only<2->{$r_0=0.5$ fm.}
\end{itemize}
}
\end{frame}

\begin{frame}{Scale setting \texorpdfstring{$t_0$}{t0}}
\only<1>{\begin{table}
    \centering
    \begin{tabular}{l r r r}
        \toprule
        Ensemble     & $L/a$           & $L$ $[\fm]$     & $a$ $[\fm]$     \\ \midrule
        $A$          & $24$            & $2.235(9)$      & $0.0931(4)$     \\ 
        $B$          & $28$            & $2.214(10)$     & $0.0791(3)$     \\ 
        $C$          & $32$            & $2.17(1)$       & $0.0679(3)$     \\ 
        $D_1$        & $32$            & $1.530(9)$      & $0.0478(3)$     \\ 
        $D_2$        & $48$            & $2.29(1)$       & $0.0478(3)$     \\ 
        \bottomrule
    \end{tabular}
\end{table}
}
\only<2->{\begin{table} % Not needed since I present the continuum extrapolation in the next figure.
    \centering
    \begin{tabular}{l r r r}
        \toprule
        Ensemble     & $t_0$[fm$^2$]   & $t_0/a^2$       & $t_0/r_0^2$     \\ \midrule
        $A$          & $0.02780(2)$    & $3.20(3)$       & $0.11121(9)$    \\ 
        $B$          & $0.02769(2)$    & $4.43(4)$       & $0.11075(10)$   \\ 
        $C$          & $0.02775(2)$    & $6.01(6)$       & $0.11099(8)$    \\ 
        $D_1$        & $0.02779(5)$    & $12.2(1)$       & $0.1112(2)$     \\ 
        $D_2$        & $0.02794(9)$    & $12.2(1)$       & $0.1117(3)$     \\ 
        \bottomrule
    \end{tabular}
\end{table}}
\note{
\begin{itemize}
    \item <2->Extrapolation results for $t_0$, where we retrieved the exact point of intersection between $t_f^2 \expect{E}$ and $0.3$ using $N_\mathrm{bs}=500$ bootstrap fits. Extrapolating to the continuum gives us $t_{0,\mathrm{cont}}/r_0^2 = 0.11087(50)$.
\end{itemize}
}
\end{frame}

\begin{frame}{Scale setting \texorpdfstring{$t_0$}{t0}}
\onslide<1->{Extrapolations for different ensemble-combinations
\begin{table}
    \centering
    \begin{tabular}{l r r}
        \toprule
        Ensembles               & $t_{0,\mathrm{cont}}/r_0^2$   & $\chi^2/\mathrm{d.o.f}$ \\ \midrule
        $A$, $B$, $C$, $D_2$    & $0.11087(50)$                         & $7.51$ \\
        $B$, $C$, $D_2$         & $0.1115(3)$                           & $0.41$ \\
        $A$, $B$, $C$, $D_1$    & $0.1119(6)$                           & $0.88$ \\
        \bottomrule
    \end{tabular}
\end{table}}
\note{
    \begin{itemize}
        \item Notice the $\chi^2/\text{d.o.f.}$ of the extrapolation versus the two other extrapolations.
    \end{itemize}
}
\end{frame}

\begin{frame}{Scale setting \texorpdfstring{$w_0$}{w0}}
\only<1>{Can also set a scale using the derivative which offers more granularity for small flow times,
\begin{align*}
    &W(t)|_{t=w_0^2} = 0.3, \\
    &W(t) \equiv t_f \frac{\dd}{\dd t_f} \left\{t^2_f \expect{E}\right\}.
\end{align*}
First presented by \citet{borsanyi_high-precision_2012}.
}
\onslide<2->{\begin{table}
    \centering
    \begin{tabular}{l r r}
        \toprule
        Ensembles               & $w_{0,\mathrm{cont}}$[fm]  & $\chi^2/\mathrm{d.o.f}$ \\ \midrule
        $A$, $B$, $C$, $D_2$    & $0.1695(5)$               & $7.12$ \\
        $B$, $C$, $D_2$         & $0.1702(3)$               & $0.53$ \\
        $A$, $B$, $C$, $D_1$    & $0.1706(6)$               & $0.86$ \\
        \bottomrule
    \end{tabular}
\end{table}}
\onslide<3->{Comparable to \citet{borsanyi_high-precision_2012} which included dynamical fermions, with $w_{0,\mathrm{cont}}=0.1755(18)(04)$ fm.}
\end{frame}

\begin{frame}{Autocorrelation in the energy}
\begin{center}
    \includegraphics[trim={0.2cm 0.0cm 0.2cm 1.0cm},clip,scale=0.6]{../../LQCD/LatticeAnalyser/figures_b645_32xx4_full/data11/post_analysis/energy/post_analysis_energy_bootstrap_autocorr.pdf}
\end{center}
\note{The autocorrelation of the energy. A value of $\tau_\mathrm{int}=0.5$ indicates that we have zero autocorrelation.}
\end{frame}


\end{document}